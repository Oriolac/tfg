% !TeX spellcheck = ca
\documentclass{article}
\usepackage[utf8]{inputenc}

\usepackage{latexsym}
\usepackage{float}
\usepackage[utf8]{inputenc}
\usepackage[catalan]{babel}
%\usepackage[english]{babel}
\usepackage{microtype}
\usepackage[hyphens]{url}
\usepackage{hyperref}
\usepackage{graphicx}
\usepackage{makeidx}
\usepackage{datetime}
\usepackage{multicol}
\usepackage{setspace}
\usepackage{enumerate}
\usepackage{booktabs}
\usepackage{braket}
\usepackage{listings}
\usepackage{color}
\usepackage{amsmath}
\usepackage{amssymb}
\usepackage[table,xcdraw]{xcolor}
\usepackage{graphicx}
\usepackage{listings}
\usepackage{hyperref}
\usepackage{vmargin}
\usepackage{wrapfig}
\usepackage{subfiles}
\usepackage{float}
\usepackage{amsmath}
\usepackage{amssymb}
\usepackage{tikz-cd}
\usepackage{multirow}
\usepackage{pgffor}
\usepackage{enumitem}
\usepackage{iflang}
\usepackage{varioref}
\usepackage{hyperref}
\usepackage{cleveref}

%%%%%%%%%%%%%%%%%%%%%%%%%%%%%%%%%%%%%%%
%%%%%%%%%%%% UTIL COMMANDS %%%%%%%%%%%%  

\newcommand{\nc}{\newcommand}
\nc{\supindex}{\textsuperscript}
\renewcommand{\baselinestretch}{1.5}

%%%%%%%%%%%%%%%%%%%%%%%%%%%%%%%%%%%%%%%
%%%%%%%%%%%%% CONFIG FILE %%%%%%%%%%%%%

\nc{\mytitle}{Implementació d'un sistema criptogràfic per l'enviament del consum en sistemes de comptadors intel·ligents}
\nc{\mysubtitle}{Bayesian Inference}
\nc{\authors}{Oriol Alàs Cercós}
\nc{\datetime}{15\supindex{th} of May, 2020}
\nc{\assignatura}{Treball Final de Grau}
\nc{\professorat}{Francesc Sebé Feixas}

% Per separar professors, utilitzar ','
% 	Ex: Maria, Joan, Pere

%%%%%%%%%%%%%%%%%%%%%%%%%%%%%%%%%%%%%%%
%%%%%%%%%%%%%  LANGUAGE   %%%%%%%%%%%%%

\newcommand{\tr}{\IfLanguageName{english}}

%%%%%%%%%%%%%%%%%%%%%%%%%%%%%%%%%%%%%%%
%%%%%%%%%%%%%%%%% MATH %%%%%%%%%%%%%%%%

\nc{\prob}[1]{P({#1})}
\nc{\probl}[2]{P({#1}|{#2})}

%%%%%%%%%%%%%%%%%%%%%%%%%%%%%%%%%%%%%%%
%%%%%%%%%%%%% TREE CREATOR %%%%%%%%%%%%

\setpapersize{A4}

\author{Oriol Alàs Cercós}
\date{29 d'Abril del 2019}

\def\contentsname{Índex}
\begin{document}
	
	
	\begin{titlepage}
		\begin{figure}[htb]
			\begin{center}
				\includegraphics[width=5cm]{imgs/UDL.png}
				\vspace*{\stretch{1.0}}
				\\
				\medskip
				\begin{center}
					\medskip\bigskip\bigskip\bigskip
					
					\huge\textbf{\mytitle}
					\\\medskip 	\Large  
					
					
					\bigskip\bigskip\bigskip
					\bigskip
					\normalsize{\tr{Made by}{Realitzat per:}}
					\\
					\large\textit{\authors}
					\\
					\setlength{\parskip}{1em}
					\normalsize{\tr{Delivery}{Data de lliurament:}}
					\\
					\large{\datetime}
				\end{center}
				\vspace*{\stretch{2.0}}
			\end{center}
		\end{figure}
		\begin{flushright}
			Universitat de Lleida
			\\
			Escola Politècnica Superior
			\\
			Grau en Enginyeria Informàtica
			\\
			\assignatura
			\\
			\medskip
			\textbf{\tr{Professorate:}{Tutor:}}
			\\
			\foreach \n in \professorat{\n\\}
		\end{flushright}
		\thispagestyle{empty} 
	\end{titlepage}
	\tableofcontents
	\thispagestyle{empty} 
	%\newpage
	\listoffigures
	\listoftables
	\thispagestyle{empty}
	\newpage
	\section{Objectius}
	\begin{itemize}
		\item Introducció al problema de la privacitat del enviament de consum dels comptadors intel·ligents. Explicar diferents sol·lucions ja realitzades \cite{busom}.
		\item Estudi del protocol criptogràfic \cite{recsi} per l'enviament de dades del consum en sistemes de comptadors intel·ligents.
		\item Introducció a la criptografia usada pel protocol.  \cite{diffie-hellman} \cite{elgamal} \cite{kangaroo} \cite{pollard}
		\item Programació del prototip en Java 8 amb l'ús de la llibreria cig-lib. \cite{smart}
		\item Pujar la llibreria cig-lib a maven.
		\item Anàlisis del rendiment del sistema en funció del nombre de comptadors i dels diferents algoritmes que es poden realitzar.
	\end{itemize}
	\section{Índex de la memòria}
	\begin{enumerate}
		\item \textbf{Introducció}. S'introduirà el problema i s'explicarà el ritme del projecte i què s'explica en cada part.
		\item \textbf{Criptografia i teoria de la computació}. S'explicarà el problema en que es basa la solució per tal de poder protegir la privacitat i també s'explicaran les diferents propietats o aplicacions criptogràfiques que s'utilitzaran.
		\begin{enumerate}
			\item Encriptació  simètrica i asimètrica
			\item Problema del logaritme discret
			\begin{enumerate}
				\item Classificació
				\item Diffie-Hellman
				\item ElGamal
				\item Pollard's Lambda
			\end{enumerate}
			\item Encriptació homomòrfica
		\end{enumerate}
		\item \textbf{Propostes}. Es començarà amb una petita introducció de les propostes ja fetes classificades en tres tipus. S'endinsarà en les propostes de Busom i Recsi que són les que es basa el projecte.
		\begin{enumerate}
			\item Solucions pertorbatives
			\item Solucions anònimes
			\item Solucions agregatives
			\begin{enumerate}
				\item Proposta Busom
				\begin{enumerate}
				
						\item Establiment de claus.
						\item Transmissió del consum
					
					\item Problema amb la privacitat de les lectures
				\end{enumerate}
				\item Proposta Recsi
				\begin{enumerate}
					\item Configuració
					\item Establiment de claus
					\item Transmissió del consum
				\end{enumerate}
			\end{enumerate}
		\end{enumerate}
		\item \textbf{Disseny de la implementació}. S'explicarà el disseny de la implementació del projecte, com s'usa la llibreria de criptografia de CiG i es documentarà tot allò que es consideri oportú.
		\begin{enumerate}
			\item Connexió
			\begin{enumerate}
				\item Data Transfer Objects
				\item Sender i Receiver
			\end{enumerate}
			\item Protocol
			\begin{enumerate}
				\item Patró màquina estat
			\end{enumerate}
		\end{enumerate}
		\item \textbf{Anàlisis}. S'analitzaran els costos experimentals del sistema en funció del nombre de comptadors i dels algoritmes emprats per realitzar el logaritme discret. S'analitzarà on està el coll d'ampolla i quines solucions es poden donar.
		\begin{enumerate}
			\item Algorismes de computació del logaritme discret
		\end{enumerate}
	\end{enumerate}
	\section{Tasques a realitzar}
	\begin{itemize}
		\item Crear una introducció a la criptografia emprada al protocol.
		\begin{itemize}
			\item Encriptació simètrica i asimètrica.
			\item Problema del logaritme discret.
			\item Diffie-Hellman.
			\item ElGamal.
			\item Encriptació homomòrfica.
		\end{itemize}
		\item Estudiar les diferents propostes.
		\begin{itemize}
			\item Busom.
			\item Recsi.
		\end{itemize}
		\item Estudiar els diferents algoritmes per realitzar el logaritme discret i veure el seu ús de memòria i temps.
		\begin{itemize}
			\item Pollard's Lambda o Kangaroo.
			\item Brute Force logarithm.
			\item Cached Brute Force logarithm.
		\end{itemize}
		\item Realitzar un disseny del prototip.
		\item Implementar el protocol usant Java 8 i amb l'ajuda de la llibreria cig-lib.
		\begin{itemize}
			\item Aprenentatge d'ús de la llibreria cig-lib \cite{ciglib}.
			\item Realitzar la documentació  i memòria de la implementació explicant el disseny i les decisions preses.
		\end{itemize}
	\end{itemize}
	\section{Planificació temporal}
	\begin{itemize}
		\item \textbf{Setmana 21 Març.}
		\begin{itemize}
			\item Acabar introducció a la criptografia.
			\item Explicar Pollard's Lambda.
			\item Explicar Diffie-Hellman.
		\end{itemize}
		\item \textbf{Setmana 28 Març.} 
		\begin{itemize}
			\item Acabar d'explicar la proposta de Busom i acabar de fer el diagrama de comunicació. 
			\item Acabar d'explicar la proposta de Recsi, fer els diferents diagrames de seqüència.
		\end{itemize}
		\item \textbf{Setmana 4 Abril}. Realitzar els diagrames de classes del disseny de la implementació.
		\item \textbf{Setmana 11 Abril}. Realitzar el diagrama de seqüència entre comptador i subestació en tots els protocols.
		\item \textbf{Setmana 18 Abril}. Explicar el disseny de la proposta en la documentació.
		\begin{itemize}
			\item Datagrames DTO de recsi i busom.
			\item Classes amb responsabilitats en la connexió.
		\end{itemize}
		\item \textbf{Setmana 25 Abril}.Explicar el disseny de la proposta en la documentació.
		\begin{itemize}
			\item Patró de disseny State.
		\end{itemize}
		\item \textbf{Setmana 2 Maig}. Anàlisis del sistema en costos teòrics.
		\item \textbf{Setmana 9 Maig}. Anàlissi del sistema en costos experimentals.
		\begin{itemize}
			\item En funció del nombre de comptadors.
			\item En funció dels algoritmes per realitzar el logaritme discret.
		\end{itemize}
		\item \textbf{Setmana 16 Maig}. Revisar documentació i memòria.
		\item \textbf{Setmana 23 Maig}. Realitzar conclusió del projecte.
		\item \textbf{Setmana 30 Maig}. Revisar memòria i presentació.\textit{}
	\end{itemize}
	\nocite{*}
	\bibliography{document}
	\bibliographystyle{unsrt}
	
\end{document}
