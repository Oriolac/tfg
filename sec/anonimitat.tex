\begin{document}
Els sistemes d'anonimat intenten abordar el problema de la privadesa anonimitzant les dades de mesurament perquè es pugui obtenir informació, però no es pugui associar fàcilment aquesta a una persona determinada.
\\\\
En \cite{anonimity}, es proposa tenir dos identificadors per cada comptador intel·ligent, depenent de les dades que es vulguin extreure:
\begin{itemize}
	\item HFID o High-Frequency ID serà l'identificador per passar les dades que violen la privadesa de la llar. Aquestes dades són les que s'envien en una freqüència molt més alta i poden suggerir informació que vulneri la privacitat dels consumidors. Aquestes dades correspondrien al consum cada 15 o 30 minuts de la llar.
	\item LFID o Low-Frequency ID serà l'identificador que s'usarà per passar les dades de gra gruixut, és a dir, informació que es passa setmanalment o mensualment i que no pot aportar informació molt detallada sobre la llar. Aquest tipus de dades s'usa per calcular la factura elèctrica.
\end{itemize}
Per tal d'anonimitzar l'identificador d'alta freqüència, hi ha la restricció que aquest mai sigui conegut pel servei públic ni subestació, sinó per una tercera entitat\footnote{Que podria ser el fabricant dels comptadors o qualsevol altra que li hagi proporcionat coneixement} que tingui la relació sobre els dos identificadors. A causa d'això, és realment important que aquesta tercera entitat no tingui coneixement sobre les lectures dels comptadors, ja que hi hauria vulnerabilitat en la privadesa.
\\
\\
La majoria de sistemes d'anonimitat trobats en la recerca també formen part del conjunt de sistemes d'agregació de dades o tenen un sistema d'agregació de les dades per anonimitzar les lectures.
\end{document}