\begin{document}
Els sistemes pertorbatius es caracteritzen per afegir a les lectures un soroll per tal de transmetre un resultat diferencial. Encara que es realitza una modificació de la lectura del comptador per tal d'augmentar la privacitat de l'usuari, en \cite{smart-grid-technique} s'intenta parametritzar l'error màxim permès i això permet que es controli l'error en una operació d'agregació. \cite{smart-grid-technique} es centra en tres objectius:
\begin{itemize}
	\item Poder realitzar el càlcul del consum total d'una llar donat un rang de temps determinat.
	\item Poder realitzar el càlcul del consum total de totes les llars donades un cert instant en el temps.
	\item Evitar poder mesurar el consum en un instant de temps determinat d'una llar.
\end{itemize}
Per aquest motiu, si cada llar envia el seu consum periòdicament a la subestació, pot organitzar aquests valors com una matriu, tal que la suma d'una fila és el consum total del consumidor donat un temps determinat i la suma d'una columna és el consum de totes les llars en un temps determinat. Quan el comptador vulgui privacitat en un moment sensible, es proposa enviar les lectures enmascarades de manera que:
\[\sum_{i=1}^{N} c_i \approx \sum_{i=1}^{N} (c_i + x_i)\]
on $N$ és el nombre total de mesures, $x_i$ és un valor aleatori donada una distribució probabilística i $c_i$ és una mesura de consum individual. També es pot veure:
\[\sum_{i=1}^{N} c_i = \sum_{i=1}^{N} (c_i + x_i) - e_0\]
on $e_0$ és l'error obtingut donada l'addició de nombre aleatoris, per tant, $e_0 = \sum_{i=1}^{N}x_i$.\\
En \cite{smart-grid-technique} es descriu com arribar a una bona parametrització per tal de no perdre el significat de la lectura. El problema en aquests sistemes és quan es detecta una anomalia molt gran, ja que el sistema està parametritzat per tenir un ajustament eficient en un interval de lectures.
\end{document}