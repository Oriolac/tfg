\newcommand{\java}{Java 8}
\newcommand{\mavenLarge}{Apache Maven 3.6.3}
\newcommand{\maven}{\texttt{maven} }
\newcommand{\ciglib}{\texttt{CigLib} }
\newcommand{\toml}{\texttt{Toml}}

\begin{document}
\section{Anàlisis de requisits}
Des d'un punt de vista general, al voler realitzar una simulació, els requisits del projecte no tenen la projecció d'un entorn realista, ja que l'objectiu és dur a terme un anàlisis dels costos empírics del projecte, com per exemple, veure on està el coll d'ampolla o mesurar i poder parametritzar els valors més idonis pel protocol.
\\
\\
En més concret, s'han trobat els següents requisits:
\begin{itemize}
	\item Des del punt de vista d'arquitectura de xarxa, es pot veure clarament que el model es pot definir com client/servidor, ja que la interacció entre les diferents entitats és centralitzada.
	\item És important mantenir el codi més obert possible per tal de realitzar els canvis de la manera més còmode, sobretot en els següents apartats: la connexió entre els clients i el servidor i la incorporació de diferents criptosistemes per realitzar el xifratge.
	\item S'ha de tenir en consideració el llenguatge i les eines a utilitzar per tal de facilitar la lògica.
	\item S'ha d'intentar tenir la capacitat de realitzar els canvis sobre les diferents paquets i llibreries en cas que es necessiti.
	\item Un cop creada la implementació, s'ha de poder realitzar un anàlisis de costs el més detallat possible.
	\item El sistema tant pel punt de vista del comptador com per la subestació ha de ser totalment configurable. És a dir, s'ha de passar per paràmetre la configuració del protocol explicada en la secció \ref{sec:configuracio-recsi}.
\end{itemize}
\subsection{Característiques tècniques}
La simulació del projecte s'ha implementat utilitzant \texttt{\java} i \texttt{\mavenLarge} per la gestió de paquets. La inclinació per utilitzar \texttt{\java} és degut a la llibreria \ciglib creada per Víctor Mateu, que recopila implementacions tant de sistemes criptogràfic com de funcions hash, entre altres. D'aquesta forma, s'intenta assolir el màxim els requisits detallats anteriorment.
\\
\\
A l'utilitzar \maven per la gestió de paquets i dependència, s'ha canviat la estructura de la llibreria i s'ha pujat a un repositori remot \cite{ciglib}. Per instal·lar-la només cal configurar el servidor de \texttt{GitHub} a \maven i cridar la dependència al projecte que es vulgui.
\\
\\
Per realitzar la configuració, s'ha decidit utilitzar el format \toml\footnote{Format de fitxer per a fitxers de configuració.} ja que és fàcil i ràpid de llegir i d'escriure a causa de la seva sintaxis i semàntica minimalista i està dissenyat per transformar sense ambigüitats el fitxer a un diccionari.
\section{Criptografia}
A l'hora de fer una primera iteració sobre el projecte, es va veure clar crear una capa entre \ciglib i el sistema, d'aquesta manera, s'intenta satisfer el requisit de mantenir el codi més obert possible per a possibles futurs canvis. A fi de realitzar-ho, s'han creat interfícies, les implementacions les quals aporten la lògica de xifrar i desxifrar.
\section{Connexió}
\subsection{Data Transfer Objects}
\section{Protocol}
\subsection{Patró Màquina-Estat}
\subsubsection{RECSI}
\subsubsection{Busom}
\end{document}
